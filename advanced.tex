\section{Más funcionalidades útiles}
\frame
{
\frametitle{Más funcionalidades útiles}
\begin{itemize}
 \item \textbf{pull}\\ \indent
Utilizándolo sin ningún parámetro adicional -\textbf{git pull origin master}- funciona igual que un \textit{update} en Subversion. Es decir, se trae los cambios y hace \textit{merge}. Si le pasamos el parámetro \textit{-{}-rebase}, funciona como un \textit{rebase}. Es decir, se trae los cambios hasta el último commit conocido antes de nuestros cambios locales y luego aplica estos cambios locales encima.
 \item \textbf{gitignore}\\ \indent
Este fichero se suele colocar en la raíz del proyecto y el contenido suele ser un listado de elementos que no queremos que sean reconocidos como ficheros del repositorio. 
 \item update-index -{}-assume-unchanged
 \item reset (soft + hard)
\end{itemize}
}

\section{Algo más avanzado}
\frame
{
\frametitle{Algo más avanzado}
\begin{itemize}
 \item squash
 \item format-patch
 \item cherry-pick
 \item submodules
\end{itemize}
}
