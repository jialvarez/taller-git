\section{Tags}
\frame
{
\frametitle{Tags}
\begin{itemize}
 \item En SVN, comúnmente se congela una versión copiando y pegando el \textit{trunk} a la carpeta \textit{tags} y subiendo esta nueva carpeta
 \item En Git, un tag no es más que una \textbf{etiqueta} en un commit concreto
 \item Con este simple acto, Git puede \textbf{recuperar una versión} en cualquier momento utilizando la etiqueta del tag
 \item Dos tipos:
 \begin{itemize}
  \item \textbf{Lightweight tags:} git \textbf{tag} <etiqueta>
  \item \textbf{Annotated tags:} git \textbf{tag} -a <etiqueta> -m <mensaje>
 \end{itemize}
 \item Para recuperar un tag, basta con hacer git \textbf{checkout} <tag>

\end{itemize}
}
